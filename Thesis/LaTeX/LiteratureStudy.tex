\mainmatter
\pagestyle{headings}

\chapter{Literature Study}
\label{ch:literature-study}

\section{What is a chatbot?}
A chatbot is a computer software program which combines \acrfull{nlp}, \acrfull{nlu} , \acrfull{nlg} \citep{Adamopoulou2020}, \acrfull{ai} and \acrfull{ml} algorithms to simulate a human-like behaviour. It can respond to questions and hold a conversation. This type of technology is what is called \acrfull{hci} \citep{Adamopoulou2020}.  \acrshort{hci} is most likely to become the most widely researched topic in the \acrshort{ai} community \citep{Bansal2018}. Chatbots are used in a variety of sectors such as education, health and of course e-commerce.\\
\break
The definition given by the dictionary goes as follows: “A computer program designed to simulate conversation with human users, especially over the Internet”\citep{Lexico2022}. Some more technical definitions are:
\begin{enumerate}
	\item A chatbot is an artificial intelligence program and a \acrfull{hci} model \citep{Bansal2018}.
	\item A chatbot is a software which act as tool that able to interact with human using human’s natural language with aim to assist human daily task \citep{Muizzah2021}.
	\item A chatbot  is  a  computer  program  which  responds  like  an  intelligent  entity when  conversed  with \citep{Khanna2015}.
\end{enumerate}
\break
Multiple types of chatbots exist, going from text-based to voice-based to a combination of both \citep{Radziwil2021}. Below is a more detailed taxonomy of chatbots.\\

\subsection{Taxonomy of a chatbot}
There are many variants of chatbots such as digital assistants, artificial conversation entities, interactive agents and smart bots \citep{Adamopoulou2020}. In this research, the focus will be on the interactive agents. Since these are made to support customer service, this research will mostly be focusing on interactive agents.\\
\break
These variants can be further divided into different categories. Chatbots can be categorise based on multiple factors such as:\\

\subsubsection{Architecture}
This categorisation is based on the way the chatbots are constructed, their internal architecture. Two types emerge: \textbf{rule-based} chatbots and \textbf{ai-based}. Rule-based chatbots use already developed or newly handcrafted algorithms as a decision maker to identify both knowledge and response. This model is fixed and can’t handle new inputs. On the other hand, \acrshort{ai}-based chatbots are chatbots that use \acrlong{ml} algorithms to generate a response based on the data provided. They keep on learning and improving based on the previous learning models. This way they are better equipped to handle new inputs \citep{Maroengsit2019}.\\

\subsubsection{Domain}
The domain category is based on how broad the domain is in which they are used. \textbf{Domain specific} chatbots focus on \textbf{specific industries}. They have knowledge of the industry they are deployed in. On the other hand, \textbf{general domain} chatbots are created to support conversations in \textbf{open topics}. They don’t support a specific domain. Instead, they are versatile and may be applied to any discipline \citep{Maroengsit2019}.\\

\subsubsection{Platform}
Chatbots can be divided up based on the platform on which they are active. \textbf{Social media} chatbots use platforms such as Facebook Messenger, Facebook WhatsApp or Telegram Messenger for their communication with the user. \textbf{Web based} chatbots are most common on websites of companies. There, they get their own chat box to serve clients that way. \textbf{Standalone program} chatbots are installed as a separate program on the device of the user \citep*{Maroengsit2019, Xu2017, CICBA2018}.\\

\subsubsection{Type of service provided}
\textbf{Goal-based} chatbots have a \textbf{set goal} they want to reach. For example, a predefined goal could be to sell a specific product to the customer. Once the product is sold, the chatbot has successfully reached its goal and thus fulfilled its purpose. \textbf{Knowledge-based} chatbots are classified on \textbf{the knowledge they possess}. \textbf{Service-based} chatbots are divided on \textbf{the services they provide} to the customer. Response-generated based chatbots are classified based on what action it performs in response generation \citep{Nuruzzaman2018}. It is possible that e.g., a goal-based chatbot can be a service-based or knowledge-based chatbot and vice versa.\\

\subsubsection{Task}
This categorisation results in \textbf{task-oriented} and \textbf{non-task-oriented} chatbots. The latter is more \textbf{question oriented} and the first one is \textbf{more conversational} \citep{Nuruzzaman2018}. Task-oriented chatbots aim to assist the customer to complete a certain task and have a short conversation. Examples are Siri, Google Now and Alexa. Non-task-orientd chatbots focus on conversing with the customers to answer questions and for entertainment purposes. \citep{Nuruzzaman2018}\\

\subsection{Some examples of e-commerce chatbots}
\begin{enumerate}
	\item Billie by Bol.com
	\item Stylebot by Nike
	\item The Proximus in-app chatbot
	\item Domino's AnyWare by Domino's Pizza
	\item Zalando chatbot
\end{enumerate}

\section{Customer service chatbots in the telecom industry: a deeper insight}
The telecom sector is an industry where customer satisfaction is an important factor. The different companies within this sector provide the same kind of services and products which makes switching operators easy. According to the research by \citep{Quintino2019}, the use of chatbots has a significant effect on customer satisfaction by ensuring better customer service. The findings from the research of \citeauthor{Quintino1019} were applied to Portuguese telecom companies, with this paper the line is further drawn into the Belgian and Dutch telecom sector.\\

\subsection{The Belgian telecom industry: market overview}
The Belgian telecom sector is known as one of the most expensive in Europe, in the Eurostat survey conducted in 2019 it is even known as the most expensive in Europe \citep{Eurostat2020}. The services offered by the telecom operators can be divided into 2 categories: fixed internet and mobile.\\
\break
According to the 2020 annual report published by the \acrfull{bipt}, Proximus is the market leader in both fixed internet (45.3\%) and mobile (39\%). Telenet is in second place in both categories; for fixed Internet they have a market share of 35.6\%, for mobile this is 26\%. There is also a 3rd player that has been growing in recent years, Orange has a market share of 7.2\% in the fixed internet and +/- 26\% in the mobile market. \citep*{BIPT2021,VanLeemputten2021}.\\

\subsection{The Dutch telecom industry: market overview}
Just like in Belgium, the Dutch telecom industry is one of the more expensive in Europe, in the 2019 annual report released by Eurostat, the Netherlands ranked 5th \citep{Eurostat2020}. In the Netherlands, mobile and fixed internet are also among the fixed services offered by telecom providers.\\
\break
In the Telecom Monitor report of the first quarter of 2021 \citep{Acm2021}, it can be deduced how the Dutch market is divided. In terms of mobile service, KPN and T-Mobile are among the market leaders, with respective market shares between 25 and 30\%. VodafoneZiggo follows with a share between 20 and 25\%. Furthermore, the market is completed by other \acrfull{mvno} that have a share between 15 and 20\%. \citep{Acm2021}\\
\break
The distribution is different in the fixed internet market, where VodafoneZiggo is market leader with a share of between 40 and 45\%. KPN follows with a percentage between 35 and 40. T-Mobile, which mainly focuses on mobile services, has a market share that barely lies between 5 and 10\%. The market is further filled by DELTA Fiber Nederland and other providers, which both have a market share between 0 and 5\%. It is therefore striking that the distribution varies greatly depending on the type of services that are offered. \citep{Acm2021}

\subsection{Challenges in the telecom industry}
Telecom operators in general face several challenges that can cause such companies to go bust. According to research by \citeauthor{Quintino2019} \citep*{Quintino2019, Malviya2012}, the most important challenges in this sector are: determining the revenue stream, rapidly changing consumer demands. The various challenges have an overall effect on customer retention. The findings from Quintino's research are also confirmed in Joshi's research \citep{Joshi2014}, in this paper, the parameters that affect the customer experience of cellular mobile services within the telecom industry of India are investigated \citep*{Joshi2014, Quintino2019, Malviya2012}.\\
\break
As in Quintino's study, this study will look at how the customer experience can be further optimised. It will be investigated how the chatbot can respond to the different challenges within the telecom sector. The results from this study can be used by telecom providers to further adjust their services to the needs of their (potential) customers.

\section{Key Attributes of a successful customer service chatbot}
Previous research \citep*{Muizzah2021, Verkeyn2018} has pointed out which characteristics are the most important. These characteristics can be grouped into some key dimensions provided by this previous research.\\

\subsubsection{Functionality}
The most important attributes are the interpretation of requests, execution of the request, flexible interpretation, ability to maintain a discussion, activation and the number of services available \citep*{Muizzah2021, Verkeyn2018}.\\

\subsubsection{Trustworthiness}
Trustworthiness handles about the chatbot containing dependable information, containing a wide range of knowledge, the possibility of rating the chatbot, robustness to unexpected input and transparency \citep*{Muizzah2021, Verkeyn2018}.

\subsubsection{Safety / Intrusion}
This category is all about protection and the respect of privacy and making sure the user is safe from intrusion \citep*{Muizzah2021, Verkeyn2018}.\\

\subsubsection{Efficiency}
The efficiency dimension specifies the ease of use, quick replies versus free text usage, availability, accessibility and the need for an account \citep*{Muizzah2021, Verkeyn2018}.\\

\subsubsection{Graphical Appearance}
Graphical Appearance is all about the design and feel of the user interface and the use of emojis and GIFs in conversations \citep*{Muizzah2021, Verkeyn2018}.\\

\subsubsection{Humanity}
The humanity dimension shows how human-like the chatbot is. The attributes are about how real the chatbot feels, if it creates and an enjoyable interaction, if it can convey a personality and if it can read and respond to certain moods of the customer \citep*{Muizzah2021, Verkeyn2018}.\\

\subsubsection{Empathy}
Empathy is about personalised options and suggestions \citep*{Muizzah2021, Verkeyn2018}.\\

\subsubsection{Responsiveness}
Responsiveness specifies the productivity and how fast the chatbot responds \citep*{Muizzah2021, Verkeyn2018}.\\
\break
By putting these 8 dimensions through \acrfull{ahp}, 7 main \acrshort{ahp} categories are found. \acrshort{ahp} is a structured approach for navigating complex decision-making processes that involve both qualitative and quantitative considerations \citep{Radziwil2021}. Table 1 lists the plausible attributes that give the highest user satisfaction score \citep{Muizzah2021}.\\

\begin{longtable}{|l|l|}
	\hline
	\textbf{Categories} &
	\textbf{Attributes} \\ \hline
	\endfirsthead
	\endhead
	\begin{tabular}[c]{@{}l@{}}Technical\\ Functionalities\end{tabular} &
	\begin{tabular}[c]{@{}l@{}}- Sentiment analytics: Intelligent (Using \acrshort{ai}, \acrshort{nlp}, \acrshort{ml}, \\ Semantic technology\\ - User friendly\\ - Automated\\ - Simple interface\\ - Interactive (using video, audio) to cater needs from different groups\\ - Multi-lingual\\ - General ease of use\end{tabular} \\ \hline
	Efficiency &
	\begin{tabular}[c]{@{}l@{}}- Robust to manipulation of data input by user\\ - Provide proper escalation channels\\ - Quick to answer\\ - Can perform damage control easily\end{tabular} \\ \hline
	Effectiveness &
	\begin{tabular}[c]{@{}l@{}}- Interpret statements and instructions accurately\\ - Linguistic accuracy\\ - Execute desired tasks correctly\\ - Contains wide array of knowledge\\ - Able to perform simple problem solving\end{tabular} \\ \hline
	Humanistic &
	\begin{tabular}[c]{@{}l@{}}- Human-like personality\\ - Transparent and disclose the chatbot identity\\ - Able to respond correctly to user's questions\\ - Can maintain the pattern of conversation within theme\end{tabular} \\ \hline
	Ethics &
	\begin{tabular}[c]{@{}l@{}}- Trained with knowledge of culture and ethics of users\\ - Protect and respect user's privacy\\ - Sensitive to user's concern (i.e., security, social concerns)\\ - Constancy\end{tabular} \\ \hline
	Technical Satisfaction &
	\begin{tabular}[c]{@{}l@{}}- Able to convey greetings\\ - Entertaining, fun and users enjoy the conversation\\ - Able to respond to the user's mood\\ - Provide emotional information using tones and expressions\end{tabular} \\ \hline
	Accessibility &
	\begin{tabular}[c]{@{}l@{}}- Ability to detect the user's intent\\ - Meeting a vast range of needs (i.e., time of buffer,\\ text interface)\\ - Available 24/7\end{tabular} \\ \hline
	\caption{Plausible quality attributes of Chatbot \citep{Muizzah2021}.}
	\label{tab:ChatbotAttributes}
\end{longtable}

These attributes can be brought under six quality assessment categories provided by the ISO 9126-1 quality model\footnote{https://www.iso.org/standard/22749.html} as seen in \ref{tab:ChatbotAttributes}.  By doing so, the quality of any piece of software, including a chatbot, can be measured \citep{Muizzah2021}.\\
\break
Depending on the goal you want to reach, the focus on the most important characteristics will shift \citep{Radziwil2021}. Overall, accessibility and efficiency come out to be the most important dimensions of a chatbot \citep{Radziwil2021}. Both \citep{Muizzah2021} and \citep{Radziwil2021} confirm these results.\\

\begin{longtable}{|c|c|}
	\hline
	\multicolumn{1}{|l|}{\textbf{Characteristics}} & \multicolumn{1}{l|}{\textbf{Sub Characteristics}}                                                            \\ \hline
	\endfirsthead
	\endhead
	Functionality   & \begin{tabular}[c]{@{}c@{}}Suitability\\ Accuracy\\ Interoperability\\ Security\\ Functional compliance\end{tabular}           \\ \hline
	Reliability                                    & \begin{tabular}[c]{@{}c@{}}Maturity\\ Fault tolerance\\ Recoverability\\ Reliability compliance\end{tabular} \\ \hline
	Usability       & \begin{tabular}[c]{@{}c@{}}Understandability\\ Learnability\\ Operability\\ Attractiveness\\ Usability compliance\end{tabular} \\ \hline
	Efficiency                                     & \begin{tabular}[c]{@{}c@{}}Time behaviour\\ Resource utilisation\\ Efficiency compliance\end{tabular}        \\ \hline
	Maintainability & \begin{tabular}[c]{@{}c@{}}Analysability\\ Changeability\\ Stability\\ Testability\\ Maintainability compliance\end{tabular}   \\ \hline
	Portability     & \begin{tabular}[c]{@{}c@{}}Adaptability\\ Installability\\ Co-existence\\ Replaceability\\ Portability compliance\end{tabular} \\ \hline
	\caption{the ISO/IEC 9126-1 internal/external quality model \citep{ISO9126}.}
	\label{tab:ISO9126}
\end{longtable}

\section{User experiences with customer service chatbots}
\subsection{How to interpret user experiences?}
In the past, customers could only make complaints and queries through mail or through a form. From 2016, this changed with the rise of \acrshort{ai} and the transition from online social networks to mobile messaging applications such as Messenger, WhatsApp, etc. Service providers tried to respond to this change by introducing chatbots for customer service \citep{Brandtzaeg2018}.\\
\break
Customer service chatbots offer the possibility to provide support to customers in real-time and around the clock. First of all, the chatbot will interpret the problem/question of the customer. Next, it will search in its model in which way it can best help the customer. Subsequently, the chatbot will execute the chosen approach until the customer's goal is reached.\\
\break
A chatbot for customer service must be tailor-made for the consumer. In the research performed by Brandtzaeg and Følstad \citep{Brandtzaeg2018} it is mentioned that some companies push to introduce chatbots into their customer service, this then leads to a technological push. This push causes chatbots to become less qualitative and potentially lead to great frustrations for customers, therefore it is important that it contains the right features so that the consumer has a pleasant experience when coming into contact with the company. A pleasant experience in itself is difficult to measure, which is why this research will examine the various aspects of the user experience in this literature study.\\

\subsection{User’s perceptions of a chatbot}
\ul{Talking with a technology-based system}\\
According to the study "\acrshort{ai}-based chatbots in customer service and their effects on user compliance" \citep{Adam2021}, the interaction between the consumer/user and the system is fundamentally social. The consumer will still consider the chatbot as a social actor, even though it is already known that it cannot show emotions \citep{Adam2021}. In this research it can thus be considered that the consumer will interact with a chatbot in the same way as a human \citep*{Cheng2021,Ischen2020}.\\
\break
The study "Humanizing chatbots: The effects of visual, identity and conversational cues on humanness perceptions" proves otherwise. If an agent is known in advance as a chatbot, the user will biasedly evaluate whether the chatbot is of sufficient quality to meet their general expectations of machines. If the agent is marked as human, the user will evaluate the quality of the agent based on previous human interactions \citep*{Go2019,Shyam2008}.\\
\break
\ul{Efficiency and effectiveness}\\
The efficiency and effectiveness in performing a productive task is considered an important and crucial factor by consumers who are using a chatbot. The purpose of the chatbot is to make the user's life easier and more productive \citep{Brandtzaeg2018}.\\
\break
The study by \citeauthor{Skjuve2019} cites that the efficiency of the chatbot was one of the main motivators for the customers to use the chatbot, confirming once again that this quality attribute is crucial.\citep{Skjuve2019}\\
\break
\ul{Empathy}\\
The study “Exploring consumers' response to text-based chatbots in e-commerce: the moderating role of task complexity and chatbot disclosure” \citep{Cheng2021} concludes that empathetic factor of the chatbot is important for successful interactions. The empathetic factor of a chatbot makes the user's perspective more appreciated, it also makes it more responsive to the specific needs of the consumer.\\
\break
The study of \citeauthor{Agarwal2021} reaffirms that empathy has a positive influence on the interaction with the chatbot. Chatbots trained by data models that took more empathy/emotional into account responded in a similar way to human agents, in turn had a better impact on customer interaction. \citep{Agarwal2021}\\
\break
\ul{Task complexity}\\
The difficulty of a task given to the chatbot has a negative effect on consumer confidence. If the task is complex, the customer will have less confidence that the task will be completed successfully \citep{Cheng2021}.
This finding is confirmed by the research of \citeauthor{XU2020}, from the results of their experiment it was concluded that users expect humans to solve complex tasks better than an \acrshort{ai} device, such as a customer service chatbot. However, for tasks with low complexity, users expect that an \acrshort{ai} device could solve it better than humans. \citep{XU2020}\\
\break
\ul{Anthropomorphic visual cues}\\
Does a chatbot with visibly human characteristics affect user satisfaction? It is a question that plays a crucial role when researching the right user models. Anthropomorphic visual cues or visibly human features can provide a certain "humanity factor" \citep{Shyam2008}, which will make the consumer consider the chatbot more human and subsequently interact more socially \citep*{Go2019,Gong2007,Kim2012, Nowak2004}.
In addition \citeauthor{Go2019,Koda1996,Wexelblat1998} found that if a user interacts with an agent who has the same characteristics as him/herself (homogeneity), then the user will view the agent as positive. Studies have found that an agent with human characteristics is considered "better" than an agent with less human characteristics \citep*{Go2019,Koda1996,Wexelblat1998}.  
\\
\break
The study of \citeauthor{Go2019} shows that message interactivity contributes to compensating for low anthropomorphic visual cue, meaning that the interactivity of a chatbot is more important to users than the visual human characteristics \citep{Go2019}. Message interactivity in a conversation can be understood as the level of contingency in message exchange. In a conversation between two people, message interactivity means understanding another's message, as well as awareness of previous conversations, if these conditions are met, we can speak of an interactive and responsive conversation \citep*{Go2019,Sudweeks1998}.\\
\break
\citeauthor{Sheehan2020} stated that chatbots with more human characteristics are more readily adopted and can therefore be considered easier to use. \citep{Sheehan2020}
\section{What makes users trust a customer service chatbot?}
Customer trust is a crucial factor in customer service, and it is important that the customer accepts the agent's feedback as qualitative information. If the customer were to question the communication with the company, this could potentially lead to the loss of potential purchases. Measuring customer trust in a chatbot has already been investigated by \citeauthor{Folstad2018}. The research provides insight into different aspects of consumer trust within customer service.
\subsection{Interpretation of question}
The fact that the chatbot may not be able to interpret the questions correctly is a possible threat to the qualitative handling of a problem/question. If the questions are misinterpreted, it is possible that the chatbot shares wrong information. According to past research, consumers are more skeptical about the feedback from a chatbot when asked a complex question \citep*{Folstad2018,Nordheim2019}.\\

\subsection{Self-presentation}
A chatbot that reveals its weaknesses and limits are considered more trustworthy by consumers. The consumer gets a better idea of what may be asked and how confidential the information may be \citep{Folstad2018}.

\subsection{Security and privacy}
The consumer expects the chatbot to meet certain security levels, especially when the chatbot handles transactions that may contain personal data. The user assumes that the service provider is responsible when the chatbot fails to support transactions \citep*{Folstad2018, Nordheim2019}.\\
\break
Since the existence of \acrfull{gdpr}, users' views on giving up personal data to machines have changed, this transition is also noticeable in chatbots. Users of chatbots want to know what happens to their data and how it is stored, they expect as little personal data as possible to be stored \citep*{Folstad2018, Nordheim2019}. 

\section{Issues and errors in the current generation of customer service chatbots}
Even though the usage of chatbots can bring forth significant advantages, they are not free of drawbacks and threats. Below are highlighted some critical issues every company should take into account.

\subsection{Privacy and data security}
Many users who interact with a chatbot are scared that their data is being captured and (mis)used. Providers should take the right precautionary measures to ensure the safety of data and to inform for what the personal data will be used. Otherwise, the usage of a chatbot will be negatively impacted \citep*{Adamopoulou2020, Duka2021, Rese2020}.

\subsection{Shortcomings of customer expectations}
One of the biggest categories of mistakes are the shortcomings of customer expectations. This varies from failing to understand their request or intent, giving wrong answers, providing a poor quality of content as well as miscommunication errors. These issues leave a bad taste in the mouth of the customer and makes them not return to the company \citep*{Adamopoulou2020, Duka2021, Nichifor2021, Sheehan2020, Margot}.

\subsection{Bias}
Bias can be shown towards multiple facets such as race, gender and religion. When taking a closer look at gender, it is interesting to see that the bias comes from both directions. Chatbots can make biased remarks towards it's customers. At the same time study shows that the customer applies gender bias towards gender identified chatbots in the field of gender-stereotypical subject domains such as mechanics for male chatbots. It is important to reduce the bias as much as possible for a smoother interaction between both parties \citep*{Adamopoulou2020, McDonnell2019}. Gender neutral bots seem to receive high satisfaction ratings in comparison to male and female chatbots, even though they are generally perceived with more negative personality traits such as colder and less friendly \citep{McDonnell2019}.

\subsection{Toxic content and social harm}
Toxic content is a severe drawback for customers. Getting insulted and getting racial slurs thrown at your head is the last thing a customer wants when interacting with a chatbot. This issue is mostly present with chatbots that use each conversation as a learning ground. Malicious users such as internet trolls can fully impact and transform the vocabulary used by the chatbot in a matter of hours \citep{Adamopoulou2020}.\\
\break
Toxic content can be taken even a step further when applied to social network bots. An example might be a chatbot that tweets about racist things, claiming that someone has committed rape or abuse, murder etc. All of this is done with the intent to hurt someone or a group of people. This is all possible because these chatbots pass the Turing Test easily \citep{Radziwil2021}.

\subsection{Lack of knowledge about good and bad attribute types}
It is difficult to tailor a chatbot to the exact need of the customers. This is due to the fact that there isn't a clear understanding about what the user deems important or critical as a success factor. The absence of knowledge on this front makes for a poor chatbot user experience \citep{brandtzaeg2020}.

\subsection{Not understanding the capabilities of the chatbot}
When customers don't fully understand the capabilities of a chatbot, he can't optimally use the service provided. This may result in frustrations or unnecessary limitations due to the absence of knowledge. It is in a company's best intrest to clearly specify what capabilities their chatbot has \citep{brandtzaeg2020}.
