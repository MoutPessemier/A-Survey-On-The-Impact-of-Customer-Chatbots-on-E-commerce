\begin{appendices}
	\chapter{Business Interviews}
	\label{ch:appendices}
	The write-out of these interview are a streamlined version. It's not a literal transcription to increase readability and because most of these interview had to be translated from Dutch to English.
	
	\section{Interview with Proximus}
	\label{in:Proximus}
	\subsection{The current state of the customer service chatbot.}
	\subsubsection{For what purposes is the conversational agent mainly used? Is it for \acrshort{faq}, to assist the help desk, deal with complaints, helping with sales or other functionalities?}
	At Proximus it was designed for support questions for residential people. That means you, me, your family, …. It’s not developed for small businesses. The goal is to help the small customers who can’t find an answer on the \acrshort{faq} pages or help centre. It also offers a new way to get requests solved in a new, innovative, fast and personalised way. That is really the first goal. Then we saw that people were also using this to ask questions about specific products or specific subscriptions. So, at some point we also had to include more sales topics. Today we really have support and sales. Our goal is to go towards the next best action, promotion, etc. But that is for the future. It’s mainly used for support.
	
	\subsubsection{And complaints?}
	People come to the bot and complain but we don’t want to offer an automatic response “okay then you go to this page and do that”. When they complain, we capture the complaint and send it to an agent. An agent is a live person.
	
	\subsubsection{What about the platform with which the customers can interact with the chatbot? Is it solely on the website or can you use it via social media, the Proximus platform or any other media?}
	We have 3 channels today. Our website, the app and Facebook Messenger. The goal is to have the chatbot at some point available through the telephone.
	
	\subsubsection{As a stand-alone application or?}
	Mainly to replace the call centers. So today, if you for example don’t have internet, you call a live agent. The goal is in the future to have intent detection directly there and have a more conversational experience rather than “press 1” or “press 2”.
	
	\subsubsection{In which languages can the chatbot be used? Dutch? French? English? Any others?}
	Dutch, French and English are available for all conversations. With our artificial intelligence, we can detect if you are talking in German or Italian and say “that’s not a language we are covering” and then propose English.
	
	\subsubsection{Why specifically these languages?}
	We live in Belgium, those are the most important languages. We also don’t have any German speaking colleagues for example, and we don’t want to Google Translate our bot. It’s also a legal requirement to have this.
	
	\subsubsection{Since Proximus is a Belgium based company, do you feel that there isn’t a need to support other languages?}
	It’s not that there is no need. It’s quite marginal. We do have expat coming to Belgium. They speak English. Our bot is really big so if we want to translate everything in another language, it means rebuilding everything and we just decided to tackle the long tale, not the short tale.
	
	\subsubsection{Do you have a rough estimate of what the \% is of different age groups that interact with the chatbot?}
	We don’t know that today. It’s not that I don’t want to provide this to you, it’s that we don’t know. I can say from manually checking conversations that is it widely spread. We have customers that are super easy with technology and that know how to use a bot. When the bot says, “can you please repeat your sentence”, that they try in an easy way to rephrase their question. We have people that don’t click on buttons and keep saying that it isn’t working. We have all of the different kind of people.
	
	\subsubsection{Do you have any idea as to why there is such a big spread?}
	Because our clients are a large group. Proximus has a position on the market to fill student needs, family needs, individual needs. We have Scarlet and Mobile Vikings. We have subscription packs for different types of families. Packs for adults, packs for students? We have pre-paid cards. It’s super broad. Our scope handles more than 450 questions a day because there are so many different questions that you can ask. It’s super broad.
	
	\subsubsection{That directly brings me to my next question. Because there are a lot of different questions, which types of questions can it answer? Only yes/no question or also open questions, very specific questions? Do we have to ask very specific questions, or can we be more vague?}
	Yes, and yes. If the question is vague, we can detect that we are missing some information and tell the customer: “okay we understand you have a question about x, here is what I can do” and then redirect. But if you say something super specific like “I forgot my pin code”, we can also provide help. Both approaches are used to not miss a conversation.
	
	\subsection{The benefits that the chatbot brings to Proximus.}
	\subsubsection{What type of services does the chatbot provide? Can it for example extend a membership or replace a pin code as you said. Or is it mainly used to redirect to the correct pages and not per se execute a task.}
	I get your question. What you must know is, once we want to be transactional, so get your pin code and not give a general answer, it needs a lot of development, and it has a impact on a lot of different teams. There is an ambition to have more transactional conversations, it’s just not feasible in the short term. So, what we can do is retrieve your pin code based on your phone number and client number, we can retrieve your bill balance to know how much you still have to pay. You can do promise of payment. It’s for people who pay later and promise to pay, we then have his promise of payment, and their service won’t be deactivated. You can ask for payment deferrals when you want to pay your bill one week later. That’s all what the bot will do in the backend as transactional conversations. We can verify the state of your subscriptions and services. I don’t know if you know but in telco, if you don’t pay your bill, you get lower speed for internet. So, when people complain about it, we can say “that’s because you didn’t pay your bill”. We can reset pin codes. You have a pin code for your phone, one for your tv. That’s something we can all reset in the backend. That’s our big bucket for now what we can do.\\
	Our strategy is of course to provide solutions, not instructions. With instructions I mean “for that task, you can go to this page where we have an explanation for you. This is your problem, here is your solution.”
	
	\subsubsection{Since you described the whole transactional conversation. Which conversation type right now is the most occurring? Transactional or informational?}
	There is less transactional conversation because we are covering more than 450 topics, we need to decide from a strategy point of view, what do we want to ‘transactionalise’, what is best for the customer depending on the channel. With a chat on the website, it is easier to say “for that you need to order that” via a webform because it was made for that interface. For some questions, we know that the website cannot help, that the \acrshort{faq} nor helpdesk can help, the self-service is not available, so we need an agent. Either we do a transaction or guide them to the best place to do it or we explain as best we can how to do it. That’s sometimes also the case.
	
	\subsubsection{My next question is about the customer service employees and their workload. As you said the chatbot will not be perfect and some questions have to be answered by agents. Do you have a rough estimate how much percent of questions a chatbot cannot answer?}
	That, I’m not sure I can really share. What I can share is that we decided that some topics we are not covering. So, that’s also part of how you design the chatbot. If you have a lot of questions about something that is not your scope, that we know we decide not to cover. The thing is, today as said, 450 questions is already a lot, but you can imagine to get there, there were a lot of analyses done. Of course, there is still a lot that we are not covering but that’s also because it doesn’t make sense to putting our focus on that.
	
	\subsubsection{What about the willingness of the customers? Today there is still a lot of pushback towards the usage of chatbots from the general masses. The feeling that they can’t help you or that they never will be able to replace a real human is real. How many of your customers are willing to use the chatbot?}
	It’s a difficult number to get because you need to ask the people “would you like to use the chatbot?” and then they would say this or that. From the analysis and benchmarking we’re doing, and from my personal story, I’m working with chatbots for over five years now, and I still ask another chatbot “Can I talk to a human”. At the end of the conversation, you can rate the conversation and sometimes you can leave open feedback. We see that people are super happy when the bot was super accurate with its response. Since it is transactional, the answer is personalised. Or when the bot is really fast. For example, with a small question, then the user got his answer very quickly. Or it is an easy step process. Say you want to do something and it’s only three steps, then the bot can say do this, this and this. Redirection to the right place, so we really help the customer so that they are really happy. Those customers will be eager to retry to use the bot.\\
	\break
	What really doesn’t work is when the bot doesn’t understand. So, an infinite loop with the bot saying: “sorry I don’t understand”. We try to be as human as possible, so we don’t repeat the phrase “sorry I don’t understand” twice but we vary. It depends on the state of mind of the person. If he’s in a nice state of mind, he will be patient and he will repeat his question. If he’s super angry because e.g., he didn’t pay his bills or he has a lot of trouble with paying his bills, he won’t have the patient. Also, here, the bot is not relevant to help.\\
	\break
	People that are patient and okay with technology they will use and reuse the bot. People that are not comfortable with technology or people who have a really big problem, and the bot doesn’t understand won’t reuse the bot. A bot is there to help easy and short questions. So, for everything that’s not in there, people won’t be eager to retry.
	
	\subsubsection{Since you said that people can leave rating after a certain conversation. Do you have an estimate whether they are mostly positive or negative ratings?}
	We have both. It’s either super positive or super negative but never in between. It’s human behaviour. If you are super happy you say ‘okay, that was great’. If you are super unhappy, you will take the time to say it. The thing is, the question is randomly triggered. It’s not triggered at the end of each conversation to have open feedback. That way we don’t want to overload our systems. So, you might fall on the person that would love to leave a comment but wasn’t offered the possibility.
	If the bot doesn’t understand, we will always take that feedback. If the user is super unhappy about Proximus in general, not only about the conversation, we also filter on that.
	
	\subsubsection{This perfectly bring us to the next question. I wanted to ask you if there are any negatives linked to the usage of a chatbot. For example, is there more user frustration right now or are there more complaints or is it the other way?}
	There were different steps. We first started with the live chat. So, when you were on the Proximus website you would directly have an agent. People were super happy because when they had a question, they almost immediately had an answer. Then we introduced the bot with \acrshort{nlp} that wasn’t great and super large. For example, when you said, “I want to cancel my Netflix account”, the bot understood: “Ah, okay, you have a question about Netflix. What do you want to do?” It was really not friendly. The difference between those 2 solutions was quite frustrating for the user. Today we have both super happy and super unhappy customers. In the super unhappy customer group, we have frustrated customers because the bot did not do well. That’s our fault. We also have unhappy customers because their internet is not working, they try to contact the bot and it doesn’t work and they feel like “I pay so much money for this and the service doesn’t work”. In the middle it’s difficult to say. But we also see, because now we are also experimenting with voice, that people are not yet ready to communicate with voice. It’s another interface. In the United States, they communicate with Siri and Alexa all the time but here, managing how to answer, what to answer, the length of the answer, … It’s really complicated so it will be even more frustration for the customer in the future.
	
	\subsubsection{Do you think that during the time where the bot wasn’t as fluent and as well put together as it is now the company a lot of customers? Was there a lot of customer churn because of the chatbot or do you think that having a bad chatbot did not impact the customer churn much at all?}
	To be honest, no, because today, unofficial number show that 10\% of interactions are done by the bot, 90\% are done via call. If out of the 10\% done by the bot, you have 50\% not happy about the bot then you need also to split between was it really the bot or Proximus in general etc. It’s difficult to say.\\
	\break
	You cannot push people away with the bot. Maybe a small percentage but they are mostly already leaving and use a bad customer service as another attribute to justify why they left. I’m convinced that the bot is not the element that made them decide “now I’m leaving Proximus”.
	
	\subsection{The business value of the chatbot.}
	\subsubsection{Can you give us a rough estimate of the increase in revenue that comes along with implementing a chatbot?}
	I can’t give you an estimate because there isn’t one today. A bot product is super expensive, and the \acrfull{roi} is not present yet. The \acrshort{roi} can be calculated if you look at the \gls{nps}. If we check customer experience, we could have some good revenue. But it’s not really revenue it’s an indirect impact. Putting a chatbot on a website, means that you open new channels, but they need an agent to take those conversations, so you need to spend money on call centers. In the beginning, to set it up, you will first lose money before you make money. So, the business value is really in the conversational and customer experience because when you see that people ask for their pin code and they get it, they think it’s magic and that’s what we want to reach from a customer experience point of view. 
	
	\subsubsection{Do you then know if it’s still in the negatives or if it has at least broke-even?}
	I can’t tell. Not that I don’t want to but it’s just too complicated to calculate and we need to take into account that the general market is not really eager to talk to bots. It’s an investment into the mid-future.
	
	\subsubsection{We assume that the chatbot is available 24/7. What happens can’t solve a question after working hours? For example, you ask a question in the weekend and the bot can’t solve it. Is the question then answered the next working day or how does it work?}
	That depends on a lot of elements. So, in general, if you’re on a Monday and its midnight and you ask a question that the bot cannot answer, we say: “okay our agents aren’t working right now, but they are back at their desk at 8:00 tomorrow”. If you are authenticated in the app and notifications are turned on, we say: “our agents are not working right now but one of our agents will pick up your question as soon as possible. Be sure to turn on notifications”. So, for that one we have a follow up. On messenger, there is also follow up because of the conversation history and notifications. On the website and you’re not logged in, we cannot retrieve your conversation when you leave the website. Then we cannot answer your question. There is no real ticket created in the backend to call you back. There is a law that if you call a service company and they don’t answer you in X minutes, they have to call you back. It’s difficult for us but good for the customer.
	
	\subsubsection{The chatbot was initially introduced because you wanted to create business value with it. Where there also other purposes? Did you want to create something unique, for marketing purposes or …?}
	No to be honest, we are a company. We need to either make money or avoid losing money. If you want to create a chatbot, you need to have money \acrshort{kpi}s. The goal is to avoid customers contacting us and that an agent needs to take the call or chat. It costs a lot to pay an agent. The business value is really for short and easy requests to have it automated. For longer or complex requests, we still want an agent to be available and we always want an agent to be available but for questions such as “what is my puck code?”, it’s trivial to automate this and an agent is not needed. That was the goal.
	
	\subsubsection{So, it clearly wasn’t just to jump on the hype train? Because there are a lot of companies who basically did that and jumped on the hype train and implemented a very poor customer experience chatbot. But you didn’t do that at the start?}
	I wasn’t there at the start. But you can say, what is the next trend and implement it. But if it’s just to experiment, you need to have a lot of money. Of course, our goal is to propose something nice and innovative to the customer but if you really ask from the business side, no the goal is to avoid losing money. From the customer experience side, it’s to delight our customer to offer a new channel, to be connected with the customer. But if you want to be real, at some point, it needs to bring value to the company.
	
	\subsubsection{Next to randomly asking customers about their experience, do you have another manner of knowing whether they were happy or not?}
	We do manual conversation screening. When we ask, “did this answer help you?”, the customer can already be gone but we still want to know how the bot did and if he is well functioning. Next to that we also capture the star rating at the end of each conversation. We sometimes do user tests with live users. Mostly a lot of different tests. That’s how we check if the bot is really helping the customer.
	
	\subsubsection{And with that live testing of users, are they mostly positive or rather negative?}
	It’s a bit biased because if you launch a post “please help us with our chatbot”, it’s usually people that are eager to play with technology that respond. They will be happy to test our technology. It will never be said officially, but in my opinion it’s biased. The live test is more focused on the \acrfull{ui} and fronted rather than the conversation.
	
	\subsubsection{Because it is not known how much revenue the bot created next question might be a bit tricky.}
	But you need to think in reverse. The chatbot doesn’t bring you money, it will help you spend less money. It helps saves money because you don’t have to pay an agent to solve easy questions. I’m not sure that we will ever get a bot that delivers money, or if you leave that to the website that they can order and pay via the bot. But in the end, if you want to have a correct bot, a live agent will need to be there as backup. The bot will always cost money. So if you automate and have a really nice experience, you will spend less.
	
	\subsubsection{Since it’s difficult to estimate the revenue that chatbots can bring by e.g., saving on wages. Do you think that other factors are more important to increase revenue above the chatbot or do you think that the chatbot will have a significant impact in the long run?}
	You need to check both directions. First the deflation, meaning you save costs. This will have an impact short-, mid- and long term. On the positive side where you really gain something, you have to look at the \gls{nps}. But that’s difficult to estimate what kind of gain you make. That is also if you provide a nice experience. Happy people will come back and stay.
	
	\subsubsection{The current growth that Proximus knows, how much \% of that growth can be traced back to the implementation of the chatbot. Or do you think that at this point, the chatbot doesn’t bring growth with it.}
	No, today it’s too difficult to tell. For Proximus, we launched our bot in June 2020. It was when Corona was starting. We wanted to assess the impact on the calls, to have a call deflation. But with Corona, everybody was calling because suddenly everyone was on the internet at the same time, and it wasn’t working. We couldn’t see the impact there. Today, teleworking is still a trend. We also had operational issues which had an impact on multiple departments for which people called as well. We haven’t had a calm period yet with which we can compare to before. And as said before, the 10\% vs 90\%. It’s not really comparable. Today, it doesn’t really make a huge impact. But that leaves the possibility to test and play with it and a lot of freedom compared to the call centre. It’s negative in that sense that it doesn’t have a huge impact but it’s positive in the sense that we can play around with it.
	
	\subsubsection{You told us quite a few times already that developing a chatbot costs a lot of money. We were wondering if you could give us a range of the costs.}
	I will not answer that question with figures. I will tell you what you need to take into account. You need a platform. You will pay a monthly or annual cost. Depending on the platform, it can go from thousands of euros to millions of euros a year. It depends on the platform, their name, etc. It depends on the size of the bot. If you have a super small bot that only answers 2 questions, you can get around with 500/year. If you have one like ours which answers 450 questions, it’s quite a lot. Then comes the \acrshort{nlp} layer. It can be integrated as part of the cost of the platform, or it can be a standalone cost. It’s about one cent per interaction. So, every time the \acrshort{nlp} is triggered. If you have one conversation where the user asks 5 different questions, you need to pay up 5 times. Then you have a team behind the chatbot. The brain of the bot, the frontend, the backend. The frontend can be in-house or via an agency. Configuration and conversation design is quite a rare resource. That means it’s usually consultancy. When I was a junior, it was around €400/day. Then the \acrshort{nlp} trainer and engineer that needs to work on the \acrshort{nlp}, train the \acrshort{nlp} and correct the \acrshort{nlp}. Then configuration for the brain. Backend and IT, that’s usually in-house. All these people need to be paid as well. That are most factors you have to keep in mind.\\
	\break
	It's also really hard to give you a figure because it depends on a so many different factors.
	
	\subsubsection{We can assume that it takes a big portion of the costs.}
	I forgot the contact centers. Every time the need to get contacted, it takes around €10/call.
	
	\subsubsection{We talked a bit about the workforce and about how many questions the chatbot can’t answer.
		Now we want to know the opposite. How much \% of the workforce can be replaced by a chatbot?}
	We don’t really cut our agents work because they just focus on more complex and more valuable resources. So that the backlog is not overloaded. Of course, you will need less agents but I have no idea how much. 
	
	\subsection{The future vision of the chatbot.}
	\subsubsection{We already spoke about the current focal points. What are the focal points for the (near) future? Is it about incorporating more languages or is it about the adaptability with other user groups or ...?}
	It’s a little bit confidential. What I can say is that we follow the latest trends and the evolution of the technology so at some point voice will take over. Voice is the next channel. We are looking into this. Functionality such as sentiment analysis is also hype. This is something we want to play with and check if it can improve our bot. Of course, at the end is to have a lot of deflation and automate more and more, be more powerful in our conversation from a functionality point of view. My focus is to take out buttons. For example, we have a yes/no-questions. That you can simply type ‘yes’ and we understand it and move on. This will smooth out the conversation. The goal is really to be as conversational as possible to really feel like you are not talking to a robot. We don’t want to fake it that you’re talking with a person. No more languages, that’s not something we will focus on. We also want to look into computer vision. For example, you send a picture of the bill and the bot can highlight and explain what’s on the bill. Every technology that you can integrate with a chatbot. Integration with Alexa, Google Home etc. Playing around with everything that’s coming up and connecting it to the bot.
	
	\subsubsection{Your expectations for the future. Do you want to start focussing more and the maintainability for example or adaptive learning and deep learning from previous conversations?}
	To be honest, it’s something I hear a lot from different platforms that they try to sell their platform with. A bot that does self-learning but I never saw it in reality. So, if that’s possible, everything that can be automated in terms of analysis and recommendations for improvement that would be amazing. Today it’s not the case yet. Sentiment analysis, we’ve been talking to our provider about if for about three years. Everything is in beta phase and if you try it out, it’s not working. It would be great to be able to not put human focus on that because it’s automated but I’m really questioning the availability of the technology and the added value if it’s not working properly.
	
	\subsubsection{Say, the self-learning works and is integrated with the chatbot. Aren’t you scared that some internet trolls might steer the conversation in a bad direction? Previous self-learning chatbots have had this issue that after a coordinated troll, that the chatbot started insulting and swearing. Aren’t you scared that that might happen if self-learning is applied?}
	No because today there is a huge difference between what the media and the internet say about chatbots and the reality when we are working on it is. We had an anecdote where the user asked something, and we said something completely random. But there is always human supervision and the fact that it is not self-learning today and it has a rule-based brain. So, if you tell him this is what he has to do, he will do it. He won’t deviate from the path. Today, 2022, I’m not afraid of it. There is a split to be made between what is demonstrated and talked about it the media and what is really the case.
	
	\subsubsection{I quickly wanted to pick in on what you said about the brain. You said that it is a rule-based brain. There are differences between pure \acrshort{ai}-based and rule-based brains. How does the brain work?}
	Today the \acrshort{ai} is on the \acrshort{nlp}, not on the conversation design. It’s not a decision tree but it’s not far from it either. It is not \acrshort{ai}-driven. The \acrshort{nlp} detects what the customer wants, then takes a decision and the rules will take over. This is simplified of course.
	
	\subsubsection{If I understood correctly, your bot works with Microsoft?}
	No, no. I just had the prices for Microsoft, we are not running on Microsoft.
	
	\subsubsection{How did the chatbot gain the knowledge and expertise? Was it via a domain expert, marketing studies, data analytics, etc.?}
	The thing is, chatbot resources and chatbots are quite new on the market. Especially in Belgium and Europe. So, everything that we used to take our decision includes benchmarking, marketing studies, check in with other companies, we read a lot of papers, a lot of conferences to exchange information. But it’s really new and every conference I’m going, we come in with new questions and new problems to solve. Luckily the impact is small so we can play and test around to get the answers we are looking for by ourselves. We learn a lot from our customers as well.
	
	\section{Interview with T-Mobile and Tele2}
	\label{in:T-Mobile}
	\subsection{The current state of the customer service chatbot.}
	\subsection{The benefits that the chatbot brings to T-Mobile/Tele2.}
	\subsection{The business value of the chatbot.}
	\subsection{The future vision of the chatbot.}
	
	\section{Interview with Telenet}
	\label{in:Telenet}
	\subsection{The current state of the customer service chatbot.}
	\subsubsection{For what purposes is the chatbot mainly used? Is it rather for \acrshort{faq}, is it also for complaints, to sell something or are there other examples for which the bot can be used?}
	
	I'm going to talk mainly about Telenet residential. You also have Telenet Business, they also have a chatbot. I'll just explain at a high level what they have. Telenet Business have more of a proactive chatbot that will pop up based on certain criteria. That's linked to a program, Marketo, a marketing \& leads generation platform. It does certain checks when you visit the Telenet Business website and if these checks are validated, the chatbot pops up. This chatbot is very limited at the moment. So, they have their platform, so to speak. For Telenet Residential, we use a different platform. We had many chatbots available like askHugo That was for motivating new customers to buy a Hugo subscription from us. This has been closed down since two years ago. We also had others like askPlay which has also been closed down. Currently we only have two. One is the Messenger bot, but that is not really a chatbot because it is rule-base and not \acrshort{ai}-driven. So, it's not an intelligent bot, it has a flow that has to be strictly adhered to by clicking buttons. If you type something, you fall out of the flow. You do end up with an advisor who can help you further but the chatbot itself will not respond. This is because it's linked to the platform we use now, and that 'dumb' bot is mainly focused on customer service. For example, the flow of technical questions is much more extensive than the other sections. We of course hope that the customer is helped by the bot but there is still a large percentage that ends up with an agent. But that's normal in a way because the bot doesn't know everything and isn't intelligent. But it's already something to help the customer and especially for the agent himself that he already has context from the questions asked by the chatbot. Because of this, they don't have to do these checks anymore. This makes it a lot more efficient for them because they already know what it is about when the customer comes to them, and they can skip those questions. This makes the conversations shorter and more efficient.
	There is also a subscription window. This also gets them to us, but the flow is much shorter and much more restricted. There is way less inflow and less contact. The bulk simply has a problem or a question about their subscription. So, we have not invested in that at the moment.\\
	\break
	We are now in a decision phase to switch to a new platform and build something decent. A real \acrshort{nlp} platform where we can build everything from automation instead of the predefined flows.
	
	\subsubsection{You mentioned that certain bots have been shut down. Why is that? Did they not work well or not get much response?}
	Yes, for example for the askHugo bot, Hugo is a subscription that we offered, it was the maintenance that caused us to stop the bot. Maintenance is always a big obstacle. You have to maintain a bot and for a small team that's a lot of maintenance to maintain and improve. That bot was more intelligent, because before we used Google's Dialogflow platform. That bot was good, but the inflow was too low. So, the \acrshort{roi} was too low to keep maintaining it.
	
	\subsubsection{And for the future platform, how would you go about getting more response and interaction with the new bot? Are you going to apply a marketing strategy to that?}
	Yes, we are probably going to look at the customer service-related stuff first. That's the biggest chunk of business we get. I think every company will say this, but we have a lot of cases that come in via voice, via calls and we want to help people as much as possible via chat because that cost is lower and now, we want to put a chatbot there to focus on that. The first step is to use an \acrshort{faq} bot to tackle certain matters very easily. Afterwards, we can use the chatbot to carry out other processes for the customer and so expand step by step. But first the easy things that we can take up easily.
	
	\subsubsection{So, if I understand correctly, for the time being the chatbot can only be used via social media platforms?}
	For now, we only have the bot on Messenger. We do plan to do an experiment with WhatsApp soon. That we offer a less elaborate flow within the app there. In the future with our new platform, that can be via web chat, social media like WhatsApp and Messenger, Twitter and Instagram. We could also offer in-app. Then all possibilities are possible. \\
	\break
	"I just used the app a few days ago and there you are indeed redirected to social media." Yes, the same thing happens when you go by the website.\\
	\break
	Through social media, it is an auto reply that says they are busy dealing with your question. It depends on when you contact us, but in such a message it says: "We will take your message in X number of minutes." But that is more of an auto reply like when you work with email. Because of the experiment, we will extend this to a flow.
	
	\subsubsection{Can the chatbot be used in more languages than just Dutch?}
	Currently it is available in Dutch, French and English.
	
	\subsubsection{Are you planning to add more languages?}
	Not at the moment, but on our future platform we have asked if text translation is possible and that we can communicate in that way.
	
	\subsubsection{Do you have an idea of what age groups comprise the majority of users? Is it 20 or younger, between 21 and 40, between 41 and 60 or 60+?}
	Not specifically, but the customers we now get via our social media channels are somewhat younger. Rather young people and adults up to 40. We have been offering WhatsApp for a year to a year and a half now. You first have to convince people to use WhatsApp because there are a lot of people who just call in. It also depends on what questions and what problems there are. For example, you can't send a message if you don't have internet, so it's logical that they call. When we offer it on the contact page, we want to convince them to use WhatsApp and chat and also offer them a good experience so that they come back afterwards. They are mostly up to 40 years old.
	
	\subsubsection{The chatbot must of course be able to answer different types of questions. Do you have an idea about what kind of questions the chatbot can answer? Yes-no questions, open questions, vague questions, ...}
	With our current chatbot it's not possible to ask a question yourself. You have to work with the buttons offered within Messenger to let them know what you want. If they have a question, we hope they follow the flow. The biggest chunk is technical, so it's often about the Internet or TV. Internet is the biggest category. Eventually, they end up with an agent where they can provide additional information. We see that at the first step, most people stop. Either they take the step and go with the flow, or they drop out. They also drop out when they start typing. This happens mainly when they think: "those buttons, that's not for me" or when they want to tell their story right away. That's where about 25\% drop out.
	
	\subsubsection{Within the flow, if you then decide to abandon the buttons, but the question is a yes/no question, and you type yes instead of pressing the yes button. Does the bot understand this then?}
	No, you really need to use the buttons. It is a very stupid bot, there is no intelligence behind it. In the future, of course, our new platform should be able to do this. The same with when you type "I have an internet problem" that it immediately redirects you to the step where we already talk about the internet instead of going through all those little steps.
	
	\subsubsection{So, the buttons perform some kind of slot-filling so the agent doesn't have to ask trivial questions anymore?}
	Yes, it does. We have drawn up this flow together with the agents. We can also link tags to it, which makes it even easier for the advisor to see what it is about. They know very quickly which questions have already been asked. 
	
	\subsection{The benefits that the chatbot brings to Telenet.}
	\subsubsection{What kind of services can the chatbot primarily provide? For example, being able to buy or stop a subscription, request an invoice, provide someone with their \acrfull{pin} or \acrfull{puk} code, etc.}
	The current chatbot has added value mainly for the agent with those checks, but we have also asked customers via surveys what they thought of the bot, and they are also satisfied with it. We can also see from the numbers that customers are prepared to go through that flow. That's why I mentioned the WhatsApp experiment. If we see that the customer is not responding well to that, then we will also stop that. So, the benefits are mainly for the agent but in the future, the idea is that it will also be more beneficial for the customer themselves. They can reach us at any time (24/7), so we hope that this will help them more quickly. Now it is still often the case that an agent must come to the phone, but in the future he or she will be able to do everything. So also adjusting a subscription, setting up a new one, etc. The possibilities are much more extensive than they are now. Of course, we do have to take into account what we struggle with most now. That is mainly the customer service-related issues. We should focus on that first, because that is also to minimize the cost so that it no longer goes to an agent and that agent can also focus on more important and complex matters. A chatbot won't be able to do everything, certainly not at first, but the fact that the agent doesn't have to solve all those small, trivial matters and that the chatbot can do that ensures that they can give the service that they are good at, that they can focus on that. So, it's for the advisor engagement, cost reduction, and then for the customer where we can deliver the most value. But we're not going to set up a new subscription as the first use case.
	
	\subsubsection{So, at the moment, those functionalities are not really there yet, it's more about the chatbot steering the customer in the right direction and then it's further handled by an agent?}
	At the moment, it is indeed handled by an agent and those advantages are more limited.
	
	\subsubsection{How much work pressure is relieved for an agent is perhaps not immediately applicable here since they end up with an agent anyway. It is perhaps more interesting to look at the impact of all the trivial questions that no longer have to be asked by the agent. How much time would this save an agent?}
	That is more of an operational question, I am not responsible for that. It will be a few seconds per call. Suppose 10 seconds * 50 agents, this will save some time. These are not the real figures.
	
	\subsubsection{How long does a call take on average? I estimate something like 2 minutes.}
	Maybe I should explain. Messenger works as a delayed channel. If you send us a message, the same via WhatsApp, you're not going to get an immediate response, it's not like live chat. Instead, the agent will pick up the message, within the hour for WhatsApp for Messenger it's more variable, when it has time. This process repeats per message. So it's not an instant conversation. We try to limit the number of back and forth by using the chatbot. That the agent doesn't have to ask all those basic questions at that time. We try to prevent that through the chatbot. That's the biggest metric we can look at. How many messages still happen after the chatbot with the customer. And that has gone down but not drastically with our current chatbot.
	
	\subsubsection{There has already been talk about how many \% drop out while conversing with the chatbot. Is there an overall picture about the customer's readiness to use the chatbot?}
	That's a difficult one. We do see that around 65\% of customers complete the flow from start to finish. So that 65\% are prepared to go through the flow. Of course, it's also possible that there are people among the 65\% who think: "I'll just go through the flow quickly so that I can talk to an agent". So, I can't say at the moment how many \% that is. 65\% is a good number to see that people are willing to work with the chatbot.\\
	We do have different flows and the percentages are different for each flow. But even for the longer flows, the drop-out rate is not too bad. I hope we can achieve the same figures with WhatsApp. It's mainly in the beginning that many people drop out, around 20 to 25\%. It also has a lot to do with their previous experience. Every company is on it now, chatbots are trend. But that also means that if the customer has had a previous experience and it didn't work, then the customer is very suspicious that chatbots don't work. I understand them, because I too am sometimes so frustrated by a chatbot, who isn’t. Now with us too, it's a pretty dumb bot but it's meant to only improve. In the hope that customers will get a better positive view.
	
	\subsubsection{The fact that it is between 60-65\%. Is it just due to the fact that customers are very happy with the chatbot or what do you think is the reason why this percentage is so high?}
	We once asked this in a small customer survey to get some input from the customer. The response was that they were happy to find a solution themselves with the help of the chatbot. That they can learn and find a digital solution themselves in this way. If you make a phone call, for example, then you say 'yes, I have this problem' and you are guided on the phone as to how you can solve it. But it's nice to know how the agent solved it and how I can solve it myself next time. That was a response that I found remarkable.
	
	\subsubsection{Are there other ways in which you measure customer satisfaction? Is there an open feedback moment?}
	Yes, but this is mainly done with a focus on the overall conversation. We asked a temporary extra question about satisfaction, and it was quite positive. Also, why they went through the flow. To come back to the high percentage. Customers also see it as a familiar feeling, it's a bit like the \gls{ivr}. When you call, you also have to use keys to indicate what it is about. So, it is also a way that they are familiar with the buttons. The fact that they are also addressed personally is also seen as positive. But those were the results of a temporary survey that we launched at the time the bot was launched and then launched again a year later. There was no super bad feedback, which is why we kept the bot live.
	
	\subsubsection{Can we assume that more than 50\% of the responses were positive?}
	Yes, I would even say around 70\%.
	
	\subsubsection{There is a positive side to the chatbot, but of course also a negative side. Have you noticed more frustrations from users or more complaints with the chatbot?}
	There was a certain period where something was not properly set up with the bot. This was a problem, because the customer was not helped. They had started the flow, but it didn't reach an agent. Okay, they had left the flow, but we still wanted to help them. There was bad feedback about this, but it has been resolved in the meantime. Some customers don't like chatbots. And for such people there is always frustration. "It doesn't work" or they have already tried the help the chatbot offers. Situations like that frustrate people, not everyone is happy about it, which I totally understand. We just hope to help the largest group of customers, who are a bit more positive. There are no big scenarios that come to mind.\\
	\break
	We do have to take many aspects into account if we want to switch to the new platform. Take it broadly, a bot can discriminate, be rude, etc. There are things that an agent can interpret that a bot can't feel. So, we're going to have to take that into account and be careful with that. In the future, we want to use sentiment analysis to know when the customer is angry and starts sending in caps lock or calling the bot names, then we want the bot to say, "sorry I couldn't help you, an agent is going to help you further".
	
	\subsection{The business value of the chatbot.}
	\subsubsection{Do you have an idea about the increase in sales due to the use of the chatbot and whether there is any at all?}
	I certainly don't have any hard figures, but I can say that creating the botflows was part of an entire platform that was launched. So actually, the tool itself didn't cost us that much extra because it was built into the platform. Of course, we as a team did spend time on setting up the flows. On the other hand, it did save time and efficiency for the advisor. But that's not going to be much, certainly. But it is something that allows the agents to focus on more important matters. 
	
	\subsubsection{So, it is mainly costs that have gone down?}
	Towards the agents, yes.
	
	\subsubsection{Have you noticed that fewer agents are needed since the chatbot?}
	No, that's too early. We don't see any drastic differences yet.
	
	\subsubsection{So, for the time being, it's just more efficient work?}
	Yes, it is. In the future, hopefully we will need less agents.
	
	\subsubsection{You just said that the chatbot is available 24/7. What happens if I send a message outside of opening hours?}
	During opening hours, your question will normally be answered the same day, but it will be delayed. Otherwise, it will be answered the next day. The customer does not have to send anything again, the message will be dealt with as quickly as possible.
	For WhatsApp, we have a restriction that we have to reply within the day. That's a rule from WhatsApp so that you don't use WhatsApp as a spam channel. So, the company has to send something back to the customer within the day, otherwise the chat is lost and then the customer will have to send the question again. So, we make sure that we reply within 24 hours.\\
	\break
	If we do not reply within 24 hours but later, this is considered a proactive message from Telenet, and they are marked as spam. This makes it seem as if we are contacting the customer out of the blue, even though that is not the case.
	
	\subsubsection{Has the chatbot been introduced purely for customer service purposes or also to jump on the hype train with marketing purposes?}
	The previous bots like AskPlay and AskHugo were a bit more marketing focused to promote and sell our product. Especially with AskHugo because it is aimed at young people and then it would be cool to solve that with a chatbot. That's why this chatbot used youngsters' language and cool words.\\
	The current chatbot is really fully focused on customer service because that's where we see the most benefits. That could be an asset in the future. Just look at Kate from KBC, they really use it as their digital agent who can help with everything. They market with it. That could also be an option for Telenet.
	
	\subsubsection{Do you have an estimate of how much the chatbot costs?}
	At the moment it's built into the platform so we're not paying that much for it. The AskHugo bot was a pay-as-you-go plan but I wasn't there at the start and can't give any figures either. Usually, it's everything around it that costs a lot. The implementation, the brainstorming, the maintenance behind it, ... \\
	With the choice we are now making for our future \acrshort{nlp} platform, it varies. There are some that charge per session. Here it depends on how much chat inflow you have. Higher numbers mean that you pay less per chat.
	Others work with a yearly subscription. Those are serious amounts. But the intention is that we get serious \acrshort{roi} from this, of course.\\
	For the rest, I prefer not to share the figures.
	
	\subsubsection{Can you tell us something about the biggest cost items?}
	The pay-as-you go of course, the cost of the platform. Then there is hosting and the biggest cost is for the people who help build the chatbot. You have platforms that are very low-code, where business people like me can very easily say that this flow must be made in this way. You also have platforms which are high-code and for which you still need developers to develop and set up a flow. We will probably go for a platform that provides more low-code support. For a business user, it is much easier on such platforms. Of course, you will still need developers.
	
	\subsubsection{Will you still need conversation designers, \acrshort{ai}-trainers, data scientists, etc.?}
	Yes, we are a small group now. I did some conversational design at the last one, but I am not a conversation designer. I also prefer to keep the overview. For now, we already have a data scientist, a developer, a functional analyst, an architect and a product owner. This group still needs to be expanded with a conversational designer and probably a few more developers.
	
	\subsubsection{When looking at the costs, do you mainly spend on the staff behind the chatbot or rather on infrastructure?}
	At the moment it is more on our team. The platform has not cost us anything. But to us it is also quite limited at the moment. In the future, I have my doubts. I can't really assess that right now. 
	
	\subsection{The future vision of the chatbot.}
	\subsubsection{What are the most important points of attention for the near future? I assume this is mainly the automation process for your new platform, but are there any other specific things you want to tackle?}
	Yes, once we have the platform, we can start with a good use case that answers most questions. Afterwards, we can extend this to sentiment analysis. In terms of organisation, we need a team, the platform setup, the launch of the first \acrfull{mvp}, the positioning of the bot, how we are going to offer the bot. Do we want to do this big or not, is there an avatar behind this? In short, the branding of the bot, the tone of voice, what it's going to look like. So many different aspects that we're going to focus on.
	
	\subsubsection{Are you planning to offer the chatbot proactively on the website through a chat bubble or in the app?}
	We already have such a ‘help needed’ button on certain pages. If you click on it, you get different contact options that we can configure per page. Currently, we have the options to contact us via phone or via chats via WhatsApp and Messenger. So that would be a good option to add. But we still have to decide how we want to offer that. Maybe we'll leave out all the other options and always let the bot speak first and only offer a call or live chat afterwards if the customer isn't helped any further.
	
	\subsubsection{And are you planning to add more languages? There was some talk earlier about automatic translation.}
	We have listed some use cases including that text translation. We have to see how we are going to approach this. A bit like the agile principle. For example, we did market research on the new platform that can be used for Telenet, Telenet Business and Base. Then we saw, especially with the Base customers, that they found text translation much more interesting than the customers of Telenet Residential. This is because Base focuses more on customers who do not always speak Dutch very well and then they would like to speak their own language. We will therefore tackle this use case faster for Base than for Telenet but that also depends on the priorities, effort and value we get in return.
	
	\subsubsection{Isn't it easier to make the chatbot multilingual by using the built-in languages of the platform itself?}
	We are definitely going to offer Dutch, French and English. It is a lot of work to add more languages. That would be a lot of extra work. That is why I would focus on those three languages and then use the text translation to cover the other languages. Otherwise, we have to translate all the flows into other languages. 3 languages is already more than enough work. The platform offers that and can do it perfectly, but on our side to maintain and set that up is a lot of work. So if they have such a tool, we will look at that rather than building everything again in another language.
	
	\subsubsection{We have talked about the short term with the new platform, but what about the future where is your focus there?  I heard you mentioned voice. Are there any avenues you would like to pursue in relation to chatbots?}
	Yes, you have the big categories like chatbots and then voice bots. There we see, as I said, there are a lot of people who call us, and if we could automate something there, that would already bring a lot of value on the business side. That would save a lot of costs. Now, in Belgium, I don't think we are ready for voice bots at all. I don't see that in the short term either, but in the long term. I also think it's better to start with chatbots first and then move on to voice bots later. Voice bots are also a little more complicated, asking questions differently so that the customer understands better and comes across better.\\
	Then there are also many things that we can automate on the agent side. For example, we can apply text prediction there, so that when a customer starts typing, it is automatically prefilled, like in google, which gives suggestions. Those are efficiency features that we also want to deploy. So, there are also a lot of automation use cases that we see for the agent.
	
	\subsubsection{Just to be sure, your chatbot is currently fully rule-based without \acrshort{ai}?}
	Yes, it is. I mentioned in the beginning that we have 2 chatbots. The one on Messenger, soon also on WhatsApp. The second one is the Google Assistant. We're going to stop that one soon as well, so I didn't go into about it, but we got some learnings from that one as well. So that's more the voice aspect but that was strongly linked to our entertainment package. The link with the \gls{dvr}, you would say to the Google Assistant, record this episode. This is then linked to your \gls{dvr}, and it would record the episode. But that was very specific with the link between the Google Assistant and the box. There were a few \acrshort{faq} questions, but they were very limited. We did not expand on that. There are other possibilities now with which it is best to make the link with the box. Instead of the Google Assistant, you can now do that via the Google Home. So that we are going to revalue that and lock it in to our future platform, then we could do that more extensively with more fulfilment and more towards voice bot.
	
	\subsubsection{Of course, there are different scenarios you can run in the chatbot. How do you know which scenarios to include in the chatbot. Where does this expertise come from? Is it through domain experts, data analysis, market research, etc.?}
	We have a lot of data coming in. Before the chatbot, customers were already contacting us on Messenger. We simply determined on that basis that we have a lot of technical questions (internet, TV, etc.). Then we focused on that too. That way, we first analysed the data. With that, our technical flows are more extensive, so that we can then also capture them. For example, there is a squad that came to us with a modem swap, they want to improve certain aspects of it and they see in the bot that it is not optimal, so now we are looking together at how we can do it better, what are the right questions to ask. Everything is based on data analysis.\\
	Of course, we also take into account the experience of our agents. That is also valuable information that they provide. We can't put everything in the chatbot, because otherwise you get a 40-step plan and that's not the intention. We try to pass that on as best we can or guide the customer to their solution.
	
	
	\section{Interview with KPN}
	\label{in:KPN}
	Interviewed person : Justin Wesseling (Specialist Process Design)
	\subsection{The current state of the customer service chatbot.}
	\subsubsection{For what purposes is the conversational agent mainly used? Is it for FAQ,
		to assist the help desk, deal with complaints, helping with sales or other
		functionalities?}
	I think it's a combination of exactly the things you mean or mentioned. Our chatbot is on our website and can be contacted via a chat button on the “contact” page. In principle, the customer can ask any question to the chatbot. It works that the moment we recognize the question, we will give an answer that leads to self-service if we think we can answer the question ourselves online. In the case of complaints, yes, we do not think there is a self-service option. In the event of complaints, we actually send the customer directly to live chat. So on the one hand we try to offer self-service in the bot and the second big goal is to shorten the chat time for our employees, by already asking a number of questions to the customer so that our employees don't have to do it again. to do. These are, for example, requesting data such as zip code, house number, bank account number, and things like that. For sales we also have a more proactive chatbot that we sometimes offer in our shops. You should see this as a pop-up that appears when we measure on the page that the customer has been in a shop for a little longer and in which we then say, “Hey, do you need some extra help?”. If the customer needs extra help, we will forward him/her to live chat.
	
	\subsubsection{The chatbot will not handle a purchase by itself?}
	No, that's actually because it's not technically possible yet. We currently do not have a link in our bot with the order systems, so that is not possible yet.
	
	\subsubsection{Through which platform can you communicate with your chatbot?}
	At the moment it is really only on the website. There are plans to expand that to other platforms, but that is still in the future. So it's really only on the website. Customers also have a “My KPN” app, from which customers can also start the chatbot if they are looking for answers, but that is actually also the underlying website. So it is not a separate place.
	
	\subsubsection{In which languages can the chatbot be used, is that only in Dutch or also in English?}
	That is also in English, only it is a lot less extensive. We have added a number of English sentences in our classification model and our language model, so that if a customer starts speaking English, we can also recognize this as an English-speaking customer. In that case, we do not offer the full chatbot, so we do not answer all questions that we answer in Dutch. We do have some topics that we think are common. The answers are often also somewhat generic, so for example billing information can be found here on the website; if you want to order something, you can do so on this page; if you have a malfunction, check if there is a service generic malfunction. It is small-scale, but therefore also available in English.
	
	\subsubsection{It is actually a chatbot that will steer the customers in the right direction, but will not really provide the solution itself?}
	While we give really specific answers to many topics in Dutch, we don't actually do that in English. What we often see with English customers is that they are simply looking for an employee. We then also request the data (postal code, ...) in English. It is easy to configure in the chatbot program.
	
	\subsubsection{Besides, there are no other languages?}
	No, also because there is no real need for this in the Netherlands.
	
	\subsubsection{Do you have an idea of what age groups comprise the majority of users? Is
		it 20 or younger, between 21 and 40, between 41 and 60 or 60+}
	That's not something we're actively monitoring right now. We do know that a large part of KPN customers is a somewhat older target group. But whether the chatbot use is a reflection of our "base", or whether they are younger people, we don't really have an idea at the moment. I think in terms of language we're keeping it pretty neutral, so it's not like we're really trying to target any particular age group. In principle, it is there for everyone.
	
	\subsubsection{The chatbot must of course be able to answer different types of questions.
		Do you have an idea about what kind of questions the chatbot can answer?
		Yes-no questions, open questions, vague questions, ...?}
	
	I find this a bit of a tricky question because it very much depends on what exactly customers are asking. Behind the chatbot is a classification model. We have about 150 different user intents. That is actually 150 different "boxes" in which a customer's question can fall. It could be that if a customer says, "I have an internet problem", then that falls into the box of internet problems, and then that customer will often go through a flow reasonably well. It can also be that a customer says something of which we are not entirely sure, in which case we ask a sort of control question in which we ask, "Did you mean that you have an internet malfunction?", and then we often see that the conversation goes well. Of course, there are also questions from customers that we misclassify. One of the nicest or perhaps most telling examples is when customers want to cancel a subscription or actually someone's next of kin wants to cancel a subscription. We have a separate desk for that, this is the next of kin desk. It is for example for the case of: "My grandmother has passed away and I want to cancel her subscription", then we recognise that as "cancel deceased subscription", and so then we offer the next of kin desk content. We've also had people say their mobile phone died. It's almost the same language, but then you don't want to say, "condolences" and then be sent to the next of kin desk. At that moment, you want to show the content for "my phone is broken". So in terms of language, it's sometimes very tricky. You'd rather it be like this than like that, so you don't say to the grandma, have you turned the on and off is, that's not the intention because mentally that's quite difficult. Another example that we often see is when we misclassify something about disturbances. This often happens because we think that a customer means that he had a breakdown, while this is actually not the case. "I am having a fault" is almost the same sentence as "I am not having a fault". If a customer types something and the bot then returns "I understand you have an internet outage", and if that customer then replies "I have no outage.", the bot still thinks there is an internet outage. These are challenges that we are trying to solve in the content by asking control questions and also providing opportunities for customers to click a button ("I have another question" or "I meant something else"). This might keep the conversation going because there is nothing more frustrating for customers than feeling stuck. I think that is what causes the most frustration in our customers.
	
	\subsubsection{Is there also data on the number of cases where the conversation goes wrong or a question is misunderstood?}
	That is difficult to observe, because the bot itself does not know whether it has misclassified anything. We do that a bit randomly. We are actually constantly improving the language model/classification model by adding or removing sentences, but because it is a kind of statistical model, it is also quite difficult. If you add weight to one intent and add extra sentences, then you don't always know what happens in the rest of the language model. It's a kind of water bed, where if you push more somewhere, it comes up again somewhere else. It's always a bit of a wet-finger exercise. I think that unfortunately we don't always quantify this. 
	
	\subsubsection{A moment ago, it was said that the intent is recognised and then a sort of step-by-step plan is followed to arrive at the right solution. Does your chatbot consist of \acrshort{ai} that recognises the intent and is then rules-based? Or is it fully rules-based or fully \acrshort{ai}-based? What architecture is behind that?}
	We actually buy the chatbot from another party, Nuance, which is a big company in speech recognition and things like that. It actually works like this, so we have this classification model that is built from the intentions and behind each intent, there is a dialogue. A dialogue is a separate module of the programme, therefore we have built a dialogue for each of these 150 intents. Indeed, they just go from one to two and three and four. Sometimes customers give us more information and we can skip certain steps. It is all rules-based, in the sense that the system itself is not intelligent and learns. We can recognise certain parts of sentences. For example, if the customer says, "I have a problem," we will first ask, "OK, is it about your mobile subscription or your home subscription?" because those are really two different things. If a customer says, "I have problems with my fixed Internet", then we recognise that "fixed Internet", because we have given it a sticker in the classification model for years.  The class of internet in this case is simply fixed internet, so then you can skip the first question, "Is it about mobile or is it about fixed?". You make it more relevant for a customer. We try to use all the information that customers give to the bot to make the conversation as smooth as possible.
	
	\subsection{The benefits that the chatbot brings to KPN.}
	\subsubsection{What kind of services can the chatbot primarily provide? For example,
		being able to buy or stop a subscription, request an invoice, provide someone
		with their Personal Identification Nuber (PIN) or Personal Unblocking Key
		(PUK) code, etc.}
	I think I'll come back to what I just said. There are actually two things that the chatbot can do. You have to see the chatbot as a kind of work preparer for our live chat agents. We try to make the subsequent conversation with an agent shorter by already asking for certain data. This can be personal data, but also specific steps. For example, in the event of an internet malfunction, the bot will then say: "You just need to reboot your modem". We then also show the employee that these steps have already been offered so that an employee can continue the conversation instead of doing everything all over again. On the other hand, the chatbot will guide the customer to the right page on kpn.com, so that the customer can find the answer there himself. Overall, these are the most important tasks of the chatbot.
	
	\subsubsection{Is there a call to an agent after every conversation with the chatbot, or can the chatbot handle problems on its own?}
	Yes, the second one. I think about 50\% of the calls that are started in the bot go through to a staff member. That doesn't mean that we don't solve the other 50\%, because there are also customers who drop out during the conversation. These are people who then no longer feel like it or who get stuck, but of course we try to avoid that. We do solve at least 15\% to 25\%. That's not always in the bot, because some questions we can actually answer in the bot, but not in other cases. If, for example, a remote control is broken, we can just send a new one when the customer logs on to the site. Unfortunately, the chatbot can't do that yet because we don't have that link between the bot and the backend systems. In that case, we'll redirect the bot to the KPN website, which is the cooperation the bot actually has with the online pages. In principle, customers like that won't have to go through an employee again, because they'll come in with a question like "I've got a broken remote control". We then say, "You can request a new remote control here", and most Dutch people are happy with that, at least.
	
	\subsubsection{What about the workload of customer service staff?
		What proportion of the requests is routed back to the employee?}
	It's not that about 50\% of the cases that go through to an employee have always failed in the chatbot and therefore have gone through to an employee. There are also plenty of subjects in which we, from a policy point of view, would like a customer to proceed to a staff member. For example, when cancelling a subscription, we would like to make a retention attempt. Of course, there are also subjects for which self-service is not at all adequate. For example, when it comes to complaints, we simply want to pass them on to a member of staff. 
	
	\subsubsection{Is there also a possibility to say directly to the chatbot: "I would like to call or chat live with a member of staff"?}
	It is indeed possible, but then we will first try to do the first steps of the conversation with the chatbot. For example, if a customer says in the first interaction: "I want to speak to an employee", we will say: "You want to speak to an employee, that's possible, but what is your question?". In this way, we still enter the normal chatbot. If a customer asks a different question first and then indicates halfway through the conversation "Can I speak to an employee?", then we escalate directly to an employee, so we don't make another self-service attempt. 
	
	\subsubsection{Is it also possible to find a telephone number on the site to call the employee directly?}
	You can also contact us directly by telephone. We do show a telephone number on our site, but not as the first one, it is sometimes a bit hidden. What I also mentioned is that when you call our customer service, you also have to say your questions first, and then there is a classification in a kind of intent, there we will also make a self-service attempt. 
	
	\subsubsection{Is this also with \acrshort{ai} that the speech recognition is done or is it rather manual?}
	No, that is also done by a system, but that is a different system. The customer's question is converted into text and then this is also held against another classification model, so this is the same classification model, only the supplier is different. We do want to merge this in the future, but not yet. On the phone side, it is also classified according to a certain subject. The example I just mentioned of the remote control, we can do that on the phone. We can then recognise whether there is any customer data attached based on the number the customer calls with. For example, if I call with my 06 number, then my 06 number is also registered with KPN with the info if he has a mobile subscription or a fixed subscription with television. If I were to call and say: "I want a new remote control, because it's broken", then we start a check in the \acrfull{vrs}, without you speaking to the employee, but simply with the system, to see if you qualify for a new remote control, and if so, we just send it.
	
	\subsubsection{What is the customer's willingness to use the chatbot?}
	That is quite difficult to express in figures. What we do is ask for qualitative feedback from customers. When you start a chatbot, it opens in another window and there is also a button to give feedback. What we see there is often negative feedback. I think that's a bit inherent to giving feedback, if your problem is solved, why would you give extra feedback on your user experience. If you get stuck or you're in a loop or you don't get the answer you want, then you're going to give feedback. So it doesn't say much about the overall performance. If I have to make an estimate, I think only 1/5 or 20\% would like to speak to an employee. We've also done some research with a customer lab in which we invited different customers or potential customers for a whole day and let them play with our bot. This research showed that if the bot works, customers will be very willing to solve things themselves, but if it doesn't work, there will be frustration and then they just want the employee to solve it. I do think that most of our customers who call or chat do not do so because they want to speak to an employee, but because they want a problem solved or have a question that needs answering. If a bot does this well, that's totally fine. It only becomes an issue if a bot doesn't do this well, and at that point you just want to be able to contact the employees easily. 
	
	\subsubsection{Do you have a view on the distribution between the use of the voice system and the chatbot?}
	Yes, there is a lot more calling. I think an order of 10 larger. So that means there are 10 times more phone calls than chats with the chatbot.
	
	\subsubsection{There is a positive side to the chatbot, but of course also a negative side.
		Have you noticed more frustrations from users or more complaints with the
		chatbot?}
	So what we also do, especially if customers have gone on to live chat, then after the call or a day after the call we send an email to some customers and then they can give feedback on it, and then we measure a number of things. That is the \acrshort{nps} score, for instance. The \acrshort{nps} score is the 'net promoter score', and in that we actually saw to what extent people are prepared to recommend KPN to acquaintances or friends. More important are actually the other two. That is the \acrshort{ces}, which is the "customer effort score" and the \acrshort{gcr}. That is the "Goal completion rate", this is actually the question, "did you reach your goal?" That is expressed in a percentage and that says something about how well the customer was helped. The "customer effort score" is more how much effort it took you to find the answer. What we do is compare the scores on chat and on the phone and what we see on certain topics, for example WiFi, is that the \acrshort{ces} is a lot lower on the phone than chat. It's also simply because WiFi is quite a complex subject, which can't be solved entirely via chat, but can be solved by someone on the phone. So it's better to have an employee who can guide a customer all the way through this. On that basis, we also make choices to pass on certain subjects from the chatbot to the call.  Again, it's difficult to quantify. What does frustration mean for someone? You also don't always know what customers think. We can collect a great deal of data, but it is of course very difficult to measure customer sentiment. 
	
	\subsubsection{Is there any sentiment analysis in the chatbot, for example, if someone asks a question or gives an answer using CAPS LOCK, is that also noticed?}
	We did look at it in the past to see if we could do something with it, but then we decided not to use it.
	
	\subsubsection{Is this a particular thing that you or you guys want to deal with in the future?}
	What we do in our data analytics department is to run analyses on all chat and call transcripts. When someone calls us, the conversation is actually transcribed, that goes automatically, so that is also a speech-to-text programme behind it and then \acrshort{ai} is released to distil a kind of step-by-step plan from all those conversations. You can, of course, link it to the feedback that customers give on such conversations and then you can look at this: you have a collection of conversations that went well, then you look at what was good about them and can you identify certain characteristics that apply generically to all good conversations. And the same applies to conversations that went badly: is there a kind of general denominator?
	
	\subsection{The business value of the chatbot.}
	\subsubsection{Do you have an idea about the increase in sales due to the use of the chatbot
		and whether there is any at all?}
	I don't know. In the sense of, I don't know exactly what it costs to keep the chatbot running. You would then actually have to look at what costs you take with that. Look, we buy this from a company and that costs money. Those are actually direct costs, but we also have two teams coming in to work on the bot. Yes, wouldn't you include that as a cost? So I don't really know what, I can't really put a number on it. The direct revenue is in the sense that we also save costs, of course, because fewer chats go to a customer service employee, so that, yes, to put it very directly, we need fewer customer service employees.
	
	\subsubsection{Is there any view on how many fewer people are needed than before the chatbot?}
	Yes, I find that somewhat difficult to express, but well, you could do some calculations. You can do it now with the wet finger. I estimate that we need 25\% fewer agents than before the chatbot.
	
	\subsubsection{We assume that the chatbot is available 24/7, but what happens if it cannot solve a question outside of working hours? Is the question still answered or is it postponed until the next working day?}
	Yes, the bot does, and so do our employees for malfunction questions, but we do have a technical opening hours check in our chatbot. For example, after 8pm we don't pass on invoice questions to an employee, because they have gone home, so then they have to contact us again tomorrow. 
	
	\subsubsection{So then that person gets a message of "It's out of hours, please try to send your question again tomorrow"?}
	For most questions, we will first try to point customers to the right place on kpn.com, so that they can find an answer themselves. But if, for example, they say at 3 a.m. "I want to speak to an employee", well, they're not there and that won't happen. We do offer the chatbot, but after that there is no handover to the live chat. 
	
	\subsubsection{Will these chats be kept so that they do not have to be done again?}
	No, it really has to be done again.
	
	\subsubsection{The chatbot was initially introduced because you wanted to create business value with it. Where there also other purposes? Did you want to create something unique, for marketing purposes or …?}
	I don't think that played a big part, but look what you see with a lot of companies is that a chatbot is initially deployed on one subject and then it is actually distributed from there. KPN actually chose to say from the outset, "We're going to put the chat in right across the board". That was a slightly different strategy in which we did say, or indeed in which, in the beginning, the work preparation simply asks for data so that the service employee would spend less time on it. We actually did that very well from the start, and then we gradually started adding all kinds of self-service options. We are still working on that. In the beginning, those self-service options were often referring to a place on the website, and we are now at a point where we are going to try to pull some more possibilities into bone itself by also creating access to our back-end system. That is technically a challenge. So yes, that's what we're up against, and certainly because we've already done quite a lot of optimisation in terms of content. If someone has an invoice question, at a certain point you are done explaining this and that. You find your invoice at that place, this is what it can contain. In terms of content, at a certain point you're done with that. The next step, of course, would be to be able to simply say, OK, this is your last invoice, it costs so much, and there is so much on it, and you have to pay it. But we can't do that at the moment.
	
	\subsubsection{How do you know if the customer is happy with the chatbot?}
	In the chatbot, there is the possibility for the customer to choose to give feedback, so just in the dropdown menu. If customers proceed to live chat, in most cases the employee will indicate at the end of the conversation whether the customer still wants to give feedback. This can be done via a form. We don't ask for explicit feedback if a customer has only been in the chatbot, because then we often don't know who has been in the bot. So that's a bit tricky. But if the customer has been in contact with the actual customer service, then often a request for feedback follows, a feedback form. 
	
	\subsection{The future vision of the chatbot.}
	\subsubsection{What are the most important points of attention for the near future?}
	We also want to be available in more places, so that's actually from the contact strategy. KPN wants to be where its customers are. That translates into adding more touchpoints, which may be to start from our app and chat even more easily, but also, for example, a Whatsapp or other contact options. We also want to harmonise it more from our strategy. What we actually want is that the moment a customer starts a conversation with us in Whatsapp, for example, we then know that when a customer calls, we also know that he or she has already asked a question in Whatsapp. This makes it quite channel-specific, so that we really keep the context. You can compare this to the situation where if I were to invite you to come and watch football with me tomorrow, I would say this now, and then you would be at the door tomorrow and I would say: "What are you doing here? That's how it often is in the customer service world now. We want to have that ongoing conversation with the customer and therefore keep the context. That's easy to say. It's technically difficult to do, but we're also trying to rationalise some systems, so that eventually we'll have one system where we can talk to our customers across all channels. That's actually a bit of a vision for the future.
	
	\subsubsection{What additional functions do you expect in the future? Are you going to make the chatbot more fully \acrshort{ai}-based or what are the plans?}
	That also depends a bit on what the supplier of the programme will do. We don't have much say in that. We can go there and press and say that we want this functionality, but my experience is that many chatbots are still a bit underwater, with if-then statements. I think that is still mostly rules-based rather than real \acrshort{ai} that understands and learns and develops itself.
	
	\subsubsection{Where does the knowledge/expertise for the chatbot come from? How do you know what scenarios to add to the chatbot?}
	Mainly through data analysis, and that's the advantage of using the bot across the board right from the start. We simply collected real customer questions and looked at what the most questions were about and then designed flows for the subjects that were asked the most often. We are still doing that. Every so often, we just go through a box of conversations and see if there are any topics in there that we don't have answers for.
	
\end{appendices}

